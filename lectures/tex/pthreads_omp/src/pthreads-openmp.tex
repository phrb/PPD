% Created 2021-05-11 Tue 14:35
% Intended LaTeX compiler: pdflatex
\documentclass[10pt, compress, aspectratio=169, xcolor={table,usenames,dvipsnames}]{beamer}

\usepackage{booktabs}
\mode<beamer>{\usetheme[numbering=fraction, progressbar=none, titleformat frame=regular, titleformat title=regular, sectionpage=progressbar]{metropolis}}
\usepackage{sourcecodepro}
\usepackage{booktabs}
\usepackage{array}
\usepackage{listings}
\usepackage{multirow}
\usepackage{caption}
\usepackage{graphicx}
\usepackage[english, brazilian]{babel}
\usepackage[scale=2]{ccicons}
\usepackage{hyperref}
\usepackage{relsize}
\usepackage{amsmath}
\usepackage{bm}
\usepackage{ragged2e}
\usepackage{textcomp}
\usepackage{pgfplots}
\usepackage{marvosym}
\usepackage{adjustbox}
\usepackage{tikz}
\usepgfplotslibrary{dateplot}
\definecolor{Base}{HTML}{191F26}
\definecolor{Accent}{HTML}{b10000}
\colorlet{Highlight}{Accent!18}
\setbeamercolor{alerted text}{fg=Accent}
\setbeamercolor{frametitle}{fg=Accent,bg=normal text.bg}
\setbeamercolor{normal text}{bg=black!2,fg=Base}
\usepackage{newpxtext}
\usepackage{newpxmath}
\usepackage{DejaVuSansMono}
\setmonofont{DejaVuSansMono}
\lstset{ %
  backgroundcolor={},
  basicstyle=\ttfamily\scriptsize,
  breakatwhitespace=true,
  breaklines=true,
  captionpos=n,
  commentstyle=\color{Accent},
  extendedchars=true,
  frame=n,
  keywordstyle=\color{Accent},
  rulecolor=\color{black},
  showspaces=false,
  showstringspaces=false,
  showtabs=false,
  stepnumber=2,
  stringstyle=\color{gray},
  tabsize=2,
  keywords={thrust,plus,device_vector, copy,transform,begin,end, copyin,
  copyout, acc, \_\_global\_\_, void, int, float, main, threadIdx, blockIdx,
  blockDim, if, else, malloc, NULL, cudaMalloc, cudaMemcpy, cudaSuccess,
  cudaGetLastError, cudaDeviceSynchronize, cudaFree, cudaMemcpyDeviceToHost,
  cudaMemcpyHostToDevice, const, data, independent, kernels, loop,
  fprintf, stderr, cudaGetErrorString, EXIT_FAILURE, for, dim3, pthread_t,
  pthread_create, exit, pthread_exit, long, printf, omp, parallel, private,
  default, shared, task, taskgroup, taskloop, num_tasks,
  omp_get_thread_num, omp_get_num_threads},
  otherkeywords={::, \#pragma, \#include, \#define, <<<,>>>, \&, \*, +, -, /, [, ], >, <}
}
\renewcommand*{\UrlFont}{\ttfamily\smaller[2]\relax}
\addtobeamertemplate{block begin}{}{\justifying}
\captionsetup[figure]{labelformat=empty}
\hypersetup{
colorlinks=true,
linkcolor={Accent},
citecolor={Accent},
urlcolor={Accent}
}
\makeatletter
\setlength{\metropolis@titleseparator@linewidth}{1pt}
\setlength{\metropolis@progressonsectionpage@linewidth}{2.5pt}
\makeatother

\graphicspath{{../../../ppd-images/}}

\title{Introdução a \textit{Pthreads} e \textit{OpenMP}}
\author{\footnotesize Pedro Bruel \\ {\scriptsize \emph{phrb@ime.usp.br}}}
\date{}

\begin{document}

\maketitle

\section*{Introdução}

\begin{frame}
    \frametitle{Slides}
    Os slides e todo o código fonte estão no \alert{GitHub}:

    \begin{itemize}
        \item \url{https://phrb.github.io/PPD}
    \end{itemize}
\end{frame}

\section{Motivação}

\begin{frame}{Programação Paralela: Motivação}
  \begin{columns}
    \begin{column}{0.5\columnwidth}
      \colorbox{Highlight}{Por que usar \alert{programação paralela}?}

      \vspace{0.8em}

      \alert{Desempenho}:
      \begin{itemize}
      \item Arquiteturas paralelas
      \item Memória Compartilhada
      \item SMP, hyperthreaded, multi-core, NUMA, $\dots$
      \end{itemize}

      \alert{Modelagem}:
      \begin{itemize}
      \item Descrever paralelismo natural
      \item Tarefas independentes
      \end{itemize}
    \end{column}
    \begin{column}{0.5\columnwidth}
      \begin{center}
        \includegraphics[width=.7\columnwidth]{shared_work}
      \end{center}
    \end{column}
  \end{columns}
\end{frame}

\begin{frame}{49 Anos de Tendências em Microprocessadores}
  \begin{center}
    \includegraphics[width=\columnwidth]{49_years_processor_data.pdf}
    \smaller[1]
    Fonte:
    \url{https://en.wikipedia.org/wiki/Transistor_count},
    \url{https://en.wikipedia.org/wiki/Microprocessor_chronology}
  \end{center}
\end{frame}

\begin{frame}{Um Senso de Escala: Tamanho de Microprocessadores}
  \begin{center}
    \begin{tikzpicture}
      \node[anchor=south west,inner sep=0] (image) at (0,0) {\includegraphics[width=.95\linewidth]{Comparison_semiconductor_process_nodes}};
      \node[align=center,font={\footnotesize}] at (image.south) {Fonte: \url{https://en.wikipedia.org/wiki/Microprocessor_chronology}};
    \end{tikzpicture}
  \end{center}
\end{frame}

\begin{frame}{Top500: \textcolor{NavyBlue}{\textit{\textbf{RPeak}}} e \textcolor{BrickRed}{\textit{\textbf{RMax}}}}
  \begin{center}
    \includegraphics[width=\columnwidth]{top500_rmax_rpeak.pdf}
  \end{center}
\end{frame}

\begin{frame}{Top500: Núcleos de \textcolor{NavyBlue}{\textit{\textbf{processador}}} e
\textcolor{BrickRed}{\textit{\textbf{acelerador}}}}
  \begin{center}
  \includegraphics[width=\columnwidth]{top500_accelerator_cores.pdf}
  \end{center}
\end{frame}

\section{IEEE POSIX Threads}

\subsection{Processos e Threads}

\begin{frame}
  \frametitle{IEEE POSIX Threads}

  \begin{columns}
    \begin{column}{0.5\columnwidth}

      \alert{IEEE} e \alert{POSIX}:
      \begin{itemize}
      \item Institute of Electrical and Electronics Engineers (\alert{IEEE})
      \item Portable Operating System Interface (\alert{POSIX})
      \end{itemize}

      \alert{IEEE POSIX Threads}:
      \begin{itemize}
      \item Define um \alert{modelo de execução}
      \item \alert{Independente} de linguagens
      \item Execução paralela de ``\alert{fluxos de trabalho}'' (\alert{threads})
      \item Define uma API para \alert{criação e controle} de threads
      \item \alert{Não define} detalhes de implementação
      \end{itemize}
    \end{column}
    \begin{column}{0.5\columnwidth}
      \begin{center}
        \includegraphics[width=.95\columnwidth]{process-and-thread}
      \end{center}
    \end{column}
  \end{columns}
\end{frame}

\begin{frame}{Threads: The Linux Programming Interface}
  \begin{columns}
    \begin{column}{0.5\columnwidth}
      \href{https://man7.org/tlpi}{The Linux Programming Interface}
      \mbox{(Michael Kerrisk, 2010)}

      \begin{itemize}
      \item Processos e threads
      \item Capítulos 24 a 33
      \item Detalhes sobre o kernel
      \end{itemize}
    \end{column}
    \begin{column}{0.5\columnwidth}
      \begin{center}
        \includegraphics[width=.7\columnwidth]{kerrisk_api}
      \end{center}
    \end{column}
  \end{columns}
\end{frame}

\begin{frame}
  \frametitle{API Pthreads}
  \begin{columns}
    \begin{column}{0.4\columnwidth}
      \alert{\textasciitilde{}100 funções} prefixadas por \texttt{pthread\_}:
      \begin{itemize}
      \item Gerenciamento
      \item Mutexes
      \item Variáveis condicionais
      \item Sincronização
      \end{itemize}
    \end{column}
    \begin{column}{0.6\columnwidth}
      \begin{table}[]
        \smaller[2]
        \centering
        \begin{tabular}{@{}ll@{}}
          \toprule
          \textbf{Prefixo} & \textbf{Funcionalidade} \\ \midrule
          \texttt{pthread\_} &  Gerenciamento \\
          \texttt{pthread\_attr\_} & Atributos \\
          \texttt{pthread\_mutex\_} &  Mutexes \\
          \texttt{pthread\_mutexattr\_} & Atributos de Mutexes \\
          \texttt{pthread\_cond\_} & Variáveis condicionais \\
          \texttt{pthread\_condattr\_} & Atributos de condicionais \\
          \texttt{pthread\_key\_} & Dados específicos de threads \\
          \texttt{pthread\_rwlock\_} & \textit{Locks} de leitura e escrita \\
          \texttt{pthread\_barrier\_} &  Barreiras e sincronização \\ \bottomrule
        \end{tabular}
        \caption{Algumas funções da API Pthreads}
      \end{table}
    \end{column}
  \end{columns}
\end{frame}

\subsection{Exemplos}

\begin{frame}
    \frametitle{Pthreads: Tutorial}
    \alert{POSIX Threads Programming}:
    \begin{itemize}
        \item Blaise Barney, Lawrence Livermore National Laboratory
        \item \url{https://hpc-tutorials.llnl.gov/posix/}
    \end{itemize}
\end{frame}

\begin{frame}[fragile]
  \frametitle{POSIX Threads: Hello, World!}
  \begin{adjustbox}{width=.65\textwidth,center}
    \begin{lstlisting}[basicstyle=\ttfamily\small]
      #include <pthread.h>
      #include <stdio.h>
      #include <stdlib.h>
      #define NUM_THREADS 5
      void *PrintHello(void *threadid) {
        long tid;
        tid = (long)threadid;
        printf("Hello World! It's me, thread #%ld!\n", tid);
        pthread_exit(NULL);
      }
      int main(int argc, char *argv[]) {
        pthread_t threads[NUM_THREADS];
        int rc;
        long t;
        for(t=0;t<NUM_THREADS;t++){
          printf("In main: creating thread %ld\n", t);
          rc = pthread_create(&threads[t], NULL, PrintHello, (void *)t);
          if(rc) {
            printf("ERROR; return code from pthread_create() is %d\n", rc);
            exit(-1);
          }
        }
        pthread_exit(NULL);
    }    \end{lstlisting}
  \end{adjustbox}
\end{frame}

\begin{frame}
    \frametitle{POSIX Threads: Mais Exemplos}
    Exemplos de código:

    \url{https://github.com/phrb/PPD/tree/main/lectures/tex/pthreads_omp/code_samples/pthreads}

    \begin{itemize}
        \item Hello, World!
        \item Argumentos
        \item \textit{Join}
        \item Servidor IRC: \url{https://github.com/phrb/simple-irc-server}
    \end{itemize}
\end{frame}

\section{OpenMP}

\begin{frame}
    \frametitle{OpenMP}
    \alert{Open Multi-Processing} (OpenMP):
    \begin{itemize}
        \item API para paralelismo \alert{multithreaded} e de \alert{memória
            compartilhada}
        \item \alert{Diretivas de compilador}
        \item Biblioteca de \alert{Tempo de Execução} (\alert{Runtime})
        \item Variáveis de ambiente
    \end{itemize}

    Objetivos:
    \begin{itemize}
        \item Padronizar
        \item Simplificar
        \item Promover \alert{portabilidade}
    \end{itemize}
\end{frame}

\subsection{Modelo de Programação}

\begin{frame}
    \frametitle{OpenMP: Modelo de Programação}
    \begin{itemize}
        \item Threads \alert{dinâmicas}
        \item Paralelismo \alert{explícito e aninhável}
        \item \alert{Diretivas} de compilador
        \item Modelo \alert{Fork-Join}
    \end{itemize}
\end{frame}

\begin{frame}
    \frametitle{OpenMP: Fork-Join}
    \begin{center}
        \includegraphics[width=.99\textwidth]{omp-fork-join}
    \end{center}
\end{frame}

\begin{frame}[fragile]
    \frametitle{OpenMP: Diretivas}
    Usadas para:
    \begin{itemize}
        \item Criar \alert{regiões paralelas}
        \item Distribuir \alert{blocos de código}
        \item Distribuir \alert{iterações de laços}
        \item \alert{Sincronizar threads}
        \item $\dots$
    \end{itemize}

    Modelo:
    \begin{lstlisting}[basicstyle=\ttfamily]
        #pragma omp directive [clause, ...]
    \end{lstlisting}

    Exemplo:
    \begin{lstlisting}[basicstyle=\ttfamily]
        #pragma omp parallel default(shared) private(beta,pi)
    \end{lstlisting}
\end{frame}

\begin{frame}[fragile]
    \frametitle{OpenMP: Biblioteca Runtime}
    Usada para:
    \begin{itemize}
        \item Obter e definir \alert{número de threads}
        \item Obter \alert{IDs de threads}
        \item Obter \alert{região paralela e nível de aninhamento}
        \item Obter, criar e destruir \alert{locks}
        \item $\dots$
    \end{itemize}

    Exemplo:
    \begin{lstlisting}[basicstyle=\ttfamily]
        #include <omp.h>
        int omp_get_num_threads(void)
    \end{lstlisting}
\end{frame}

\begin{frame}[fragile]
    \frametitle{OpenMP: Variáveis de Ambiente}
    Usadas para:
    \begin{itemize}
        \item Definir \alert{número de threads}
        \item Distribuir \alert{iterações de laços}
        \item Associar \alert{threads a processadores}
        \item Configurar \alert{paralelismo aninhado}
        \item Configurar \alert{threads dinâmicas}
        \item $\dots$
    \end{itemize}

    Exemplo:
    \begin{lstlisting}[basicstyle=\ttfamily]
        export OMP_NUM_THREADS=8
    \end{lstlisting}
\end{frame}

\subsection{Exemplos}

\begin{frame}
    \frametitle{OpenMP: Tutorial}
    \alert{OpenMP Programming}:
    \begin{itemize}
        \item Blaise Barney, Lawrence Livermore National Laboratory
        \item \url{https://computing.llnl.gov/tutorials/openMP}
    \end{itemize}
\end{frame}

\begin{frame}[fragile]
    \frametitle{OMP: Hello, World!}
    \begin{adjustbox}{width=.65\textwidth,center}
    \begin{lstlisting}[basicstyle=\ttfamily\scriptsize]
    #include <stdio.h>
    #include <omp.h>

    int main(int argc, char *argv[]){
        int nthreads, tid;

        #pragma omp parallel private(tid)
        {
            tid = omp_get_thread_num();
            printf("Hello World from thread = %d\n", tid);

            if(tid == 0){
                nthreads = omp_get_num_threads();
                printf("Number of threads = %d\n", nthreads);
            };
        };
        return 0;
    };
    \end{lstlisting}
    \end{adjustbox}
\end{frame}

\begin{frame}
    \frametitle{OpenMP: Mais Exemplos}
    Exemplos de código:

    \url{https://github.com/phrb/PPD/tree/main/lectures/tex/pthreads_omp/code_samples/omp}

    \begin{itemize}
        \item Hello, World!
        \item Parallel \texttt{for}
        \item Reduction
        \item Critical section
    \end{itemize}
\end{frame}

% \begin{frame}[fragile]
%     \frametitle{OpenMP: Exemplo com \texttt{taskloop}}
%     \alert{Dividir e sincronizar} $1024$ iterações de um laço entre $32$
%     \textit{threads}, usando \alert{OpenMP $\geq$ 4.5}:
%
%     \begin{lstlisting}[basicstyle=\ttfamily\scriptsize]
%     #pragma omp taskloop num_tasks(32)
%     for (long l = 0; l < 1024; l++){
%         do_something(l);
%     };
%     \end{lstlisting}
% \end{frame}

\maketitle

\end{document}
